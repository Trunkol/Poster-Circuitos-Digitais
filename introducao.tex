\section{Introdução}
\setlength{\parindent}{2cm}

Um importante cruzamento da cidade que fica entre a Avenida Açaí e a Rua Guaraná está enfrentando problemas de trânsito. Após de várias relatos de acidentes entre veículos ocorridos nesse cruzamento, o secretário de mobilidade urbana da cidade resolveu implementar um sistema de semáforos com temporização equivalente a fim de controlar o fluxo de carros. 

Estudos revelaram que alguns motoristas "apressadinhos" têm o mau costume de acelerar demasiadamente carro a fim de alcançar o sinal amarelo. Além disso, foi notado que a concentração de veículos na avenida Açaí é muito intensa nos horários de pico. De modo a contornar esses problemas, foram adicionados um medidor de velocidade na avenida Açaí e uma temporização especial que prioriza os motoristas da mesma e que deve ser ativada nos horários de maior movimento a fim diminuir descongestionamentos. O cruzamento está ilustrado na figura abaixo.

O secretário pediu aos alunos da UFRN que implementem esse sistema de semáforos em FPGA. O pagamento será uma boa nota na segunda unidade da disciplina de Laboratório de Circuitos Digitais.