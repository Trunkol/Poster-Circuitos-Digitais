\section{Introdução}
\setlength{\parindent}{2cm}
%Uma importante via de acesso entre as cidade de Natal e Parnamirim está enfrentando problemas de mobilização urbana. Este efeito é mais percebido durante o início do dia (próximo das 7h00) e final do dia (próximo às 18h00). Nestes períodos há um aumento considerável de fluxo de automóveis nos sentidos Parnamirim-Natal e Natal-Parnamirim respectivamente e durante o restante o fluxo menos intenso, mas as velocidade médias dos veículos continua igual nos dois sentidos. 

%Foi notado que um gargalo comum a estes congestionamentos ocorre próximo a um par de semáforos. Os moradores de Parnamirim que trabalham ou estudam em Natal aumentam o fluxo sentido Parnamirim-Natal e ao final do dia ocorre o efeito inverso. Resolver essa situação poderia simplificar o trânsito.

%Também foi observado que em momentos da via livre, os automóveis não respeitam a velocidade máxima permitida. Provocando muitos acidentes antes e depois desses semáforos.

%Pensando nisso, o governo das duas cidades, em parceria, contratam uma equipe de engenheiros para determinar uma temporização adequada para os semáforos e para fiscalizar a velocidade dos automóveis, foi providenciado que esses engenheiros também definam circuitos para detectam a velocidade dos automóveis através de um par de radares localizados em cada direção da via.

Um importante cruzamento da cidade que fica entre a Avenida Açaí e a Rua Guaraná está enfrentando problemas de trânsito. Após de várias relatos de acidentes entre veículos ocorridos nesse cruzamento, o secretário de mobilidade urbana da cidade resolveu implementar um sistema de semáforos com temporização equivalente a fim de controlar o fluxo de carros. 

Estudos revelaram que alguns motoristas "apressadinhos" têm o mau costume de acelerar demasiadamente carro a fim de alcançar o sinal amarelo. Além disso, foi notado que a concentração de veículos na avenida Açaí é muito intensa nos horários de pico. De modo a contornar esses problemas, foram adicionados um medidor de velocidade na avenida Açaí e uma temporização especial que prioriza os motoristas da mesma e que deve ser ativada nos horários de maior movimento a fim diminuir descongestionamentos. O cruzamento está ilustrado na figura abaixo.

O secretário pediu aos alunos da UFRN que implementem esse sistema de semáforos em FPGA. O pagamento será uma boa nota na segunda unidade da disciplina de Laboratório de Circuitos Digitais.